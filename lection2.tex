\section{Парадокс Карри}

Лямбда-исчисление было предложено Черчем в начале 1930х годов 
для формализации математики
%. Особенность лямбда-исчисления, 
%отличающего его от, допустим, обычного исчисления предикатов ---
%формализация понятия применения функций. 
В лямбда-исчислении легко 
выражаются сложные понятия --- например, натуральные числа, 
причем это достигается без введения дополнительного набора аксиом
как в формальной арифметике. Предполагалось, что эта теория
будет свободна от парадоксов в силу своей элементарности.

Однако, довольно быстро в нем нашлись неустранимые парадоксы.
Далее будет изложен один из парадоксов, это не оригинальный парадокс
1932 года --- поскольку современное лямбда-исчисление появилось в 1940
году как результат упрощения Черчем исходной теории. %Но пример 
%дает представление о проблеме.

Построим логическое исчисление на основе языка лямбда-выражений, 
добавив в паре к аппликации импликацию. 
В таком языке функция $\lambda f.\lambda x.f x \rightarrow f x$
являлась бы тавталогией. Естественно, в этом исчислении будут аксиомы
и правила вывода, как и в обычном исчислении --- среди них
будут и обычные логические аксиомы, и аксиомы про лямбда-преобразования, 
и подобным выражениям будет дан четкий формальный смысл.

Мы будем считать, что если $A=_\beta B$, то $\vdash A\rightarrow B$ и
$\vdash B\rightarrow A$.
Удивительно было бы ожидать иного --- ведь результат редукции $A$ и $B$ 
одинаков (либо обе редукции не заканчиваются). 
В программировании бывает, что мы не можем заменить, например,
вызов функции на его результат --- например, \texttt{printf("Hello, world")} нельзя 
заменить на $1$, хоть численно их результат равен.
В математике же считается, что разницы между разной записью значений нет:
уравнение $ x^2 = 4 $ выполнено и при $x = 2 \cdot \sin \pi$,
и при $x = \log_2 4$.

Также мы ожидаем доказуемость 
$$\alpha\rightarrow\alpha\rightarrow\beta\vdash\alpha\rightarrow\beta$$ и
$$\alpha\rightarrow\alpha$$ (было бы обидно, если такое нельзя было бы доказать).

%Исчисление получается очень мощным, в частности, в нем мы могли бы выразить
%правило $\alpha\rightarrow\alpha$ как одно выражение $\lambda t.t \rightarrow t$, без
%использования схем аксиом.

В таком исчислении мы могли бы ввести аксиоматику Пеано простыми
определениями (см. ):

И, не вводя никаких дополнительных аксиом, доказать, скажем,
такие утверждения:

Однако, так построенное исчисление черезчур мощно, о чем свидетельствует
следующее рассуждение.

Рассмотрим выражение $F_\alpha \equiv \lambda x. x x \rightarrow \alpha$ 
и выражение $\Phi_\alpha \equiv F_\alpha F_\alpha$.
Нетрудно видеть, что $\Phi_\alpha =_\beta \Phi_\alpha\rightarrow\alpha$.
Тогда:

\begin{tabular}{ll}
$\vdash\Phi_\alpha\rightarrow\Phi_\alpha$ & Доказуемое утверждение\\
$\vdash\Phi_\alpha\rightarrow\Phi_\alpha\rightarrow\alpha$ & бета-эквивалентность\\
$\vdash(\Phi_\alpha\rightarrow\Phi_\alpha\rightarrow\alpha)\rightarrow(\Phi_\alpha\rightarrow\alpha)$ & Доказуемое утверждение\\
$\vdash\Phi_\alpha\rightarrow\alpha$ & M.P.\\
$\vdash\Phi_\alpha$ & бета-эквивалентность\\
$\vdash\alpha$ & M.P.
\end{tabular}

Таким образом мы показали, что любое утверждение может быть выведено в данной
системе, т.е. система противоречива. Данное противоречие является следствием выразимости
в данной системе парадокса Карри. Парадокс можно продемонстрировать фразой
<<если данное высказывание истинно, то луна сделана из зеленого сыра>> или, чуть более формально,
выражением $\Phi_\alpha = \Phi_\alpha\rightarrow\alpha$ (если выражение истинно, 
то из него следует все что угодно). 

Чтобы избежать этого парадокса, находящегося в самой основе теории, мы должны чем-то поступиться.
Логично уменьшить силу лямбда-выражений, которые мы используем (например так, чтобы Y---комбинатор
был невыразим). Без остальных элементов теории обойтись будет тяжело. Из этой идеи и выросла 
современная теория типов (также, как и современная
теория множеств). Дальнейший курс во многом и будет состоять из изучения разнообразных способов
ограничения выразительной силы лямбда-исчисления, которая всё ещё способна выразить необходимые 
мысли, но уже достаточно слабых, чтобы приводить к парадоксам.

%Данная ситуация хорошо нам знакома и показывает, что механическое перенесение казалось бы 
%естественных правил в формальные условия может привести к проблемам.

%Подробнее с историей лямбда-исчисления и парадокса Карри можно познакомиться, например, в .
